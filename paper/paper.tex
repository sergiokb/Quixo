\documentclass[12pt, twoside]{article}
\usepackage{jmlda}
%\newcommand{\hdir}{.}

\begin{document}
\English

\title
	[Thesis] % short title for page headings, not necessary if a full title fits the headings
    {Teaching LLM to play Quixo} % full title
\author
	[S.\,M.~Kunin-Bogoiavlenskij] % short list of the authors (<= 3) for page headings, is necessary only if the full list does not fit the headings
	{S.\,M.~Kunin-Bogoiavlenskij, A.\,S.~Trushin} % full list of the authors, presented in the table of contetns of the issue
    [S.\,M.~Kunin-Bogoiavlenskij$^1$, A.\,S.~Trushin$^2$] % list of the authors presented in the title page of the article, is necessary only if it differs from the full list of the authors in braces, i.e. '{' and '}'
\email
    {kuninbogoiavlenskij@gmail.com; tarstars@yandex.ru}
%\thanks
%    {The research was
%     %partially
%     supported by the Russian Foundation for Basic Research (grants 00-00-0000 and 00-00-00001).}
\organization
    {$^1$Moscow Institute of Physics and Technology;
     $^2$Yandex}
\abstract
    {	
	This paper presents a study on teaching Large Language Models (LLMs) to play the board game Quixo. 
	We explore various strategies and models to determine the most effective approach for training LLMs to excel in this strategic game. 
	Our work includes implementing MinMax algorithm, leveraging MapReduce for efficient computation, and experimenting with different language models such as YandexGPT7B and torch.nn.Transformer. 
	Additionally, we investigate the impact of multimodal input, including voice commands, and evaluate different notation systems for representing game moves. 
	The findings aim to provide insights into the capabilities of LLMs in strategic gameplay and the effectiveness of various training methods.

    	\textbf{Keywords}: \emph{Large Language Models; Quixo; MinMax algorithm; MapReduce}}

%\titleRus
%    [Шаблон статьи для публикации] % short title for page headings, not necessary if a full title fits the headings
%    {Шаблон статьи для публикации в журнале <<Машинное обучение и анализ данных>>} % full title
%\authorRus
%    [И.\,О.~Автор] % short list of the authors (<= 3) for page headings, is necessary only if the full list does not fit the headings
%    {И.\,О.~Автор, И.\,О.~Соавтор, И.\,О.~Фамилия} % full list of the authors, presented in the table of contetns of the issue
%    [И.\,О.~Автор$^1$, И.\,О.~Соавтор$^2$, И.\,О.~Фамилия$^{1,2}$] % list of the authors presented in the title page of the article, is necessary only if it differs from the full list of the authors in braces, i.e. '{' and '}'
%\thanksRus
%    {Работа выполнена при
%     %частичной
%     финансовой поддержке РФФИ, проекты \No\ \No 00-00-00000 и 00-00-00001.}
%\organizationRus
%    {$^1$Организация, адрес; $^2$Организация, адрес}
%\abstractRus
%    {Данный текст является шаблоном статьи, подаваемой для публикации в журнале <<Машинное обучение и анализ данных>>.
%
%    Аннотация описывает основную цель работы,
%    особенности предлагаемого подхода и~основные результаты.
%    Сведения, содержащиеся в заглавии статьи, не должны повторяться в тексте авторского резюме.
%    В аннотации не должно быть ссылок на литературу и, по возможности, формул.
%	
%	Также необходимо представить расширенную структурированную аннотацию на английском языке объемом 200--300 слов.	
%	Английская аннотация может не быть дословным переводом русского текста и должна быть написана хорошим английским языком.
%	
%	В титульном заголовке необходимо указать полный, официально принятый, переводной вариант названия организации.
%	Указывать нужно только ту часть названия, которая относится к понятию юридического лица,
%	не вписывая названий кафедры, лаборатории или другого структурного подразделения внутри организации.
%	Необходимо указать полный юридический адрес, или, как минимум, город и страну.
% 	
% 	При выборе ключевых слов основным критерием является их потенциальная ценность для выражения содержания документа или для его поиска.
%	В качестве ключевых слов могут использоваться термины из названия, аннотации, вступительной и заключительной части текста статьи.
% 	При подборе ключевых слов рекомендуется использовать базовые термины вместе с более сложными, допускается использование повторов и синонимов.
%	Не рекомендуется использование слишком сложных слов, слов в кавычках, слов с запятыми.
%	По возможности следует применять слова в основной форме именительного падежа единственного числа.
%	Рекомендуемое количество ключевых слов~-- 5-7, количество слов внутри ключевой фразы~-- не более 3.
%	
%\bigskip
%\noindent
%\textbf{Ключевые слова}: \emph {ключевое слово; ключевое слово; еще ключевые слова, разделенные <<;>>}
%}


%these fields are filled in by the journal editors
%\doi{10.21469/22233792}
%\receivedRus{01.01.2017}
%\receivedEng{January 01, 2017}

\maketitle
%\linenumbers

\section{Introduction}

The field of artificial intelligence has witnessed significant advancements in recent years, particularly in the development of Large Language Models (LLMs). These models have demonstrated remarkable capabilities in understanding and generating human language, making them potential candidates for complex tasks such as playing strategic board games. In this paper, we focus on teaching LLMs to play Quixo, a board game that requires strategic thinking and planning. We aim to understand how clever LLMs are in comprehending the spatial characteristics and strategy involved in the game.

Our study is structured around several key objectives.
 First, we aim to implement and evaluate the performance of MinMax algorithm, which was proposed in the recent article. 
 To handle the computational complexity, we employ MapReduce to efficiently distribute and process the calculations. Second, we train LLM to play using the acquired full tree of win/loss game states.
Then we compare the learning capabilities of different language models, including YandexGPT7B (with 7 billion parameters), torch.nn.Transformer, and convolutional neural networks (CNNs). 
 This comparison will help us understand which models are best suited for learning the intricacies of Quixo.

Additionally, we explore the potential of multimodal input, specifically voice commands, to enhance the interaction and learning process of the LLMs. 
We also investigate various notation systems for representing game moves, including different visual and verbal methods, to determine the most effective approach for training and communication.

Through this study, we aim to contribute to the understanding of LLMs' capabilities in strategic gameplay and provide insights into effective training methods and input modalities.
%
%\section{Preparing a manuscript}
%\noindent
%Manuscripts are prepared using \verb'jmlda.sty' style package.
%You are recommended to use \verb'jmlda_rus.bst' and \verb'jmlda_eng.bst' style files for generating bibliography using Bib\TeX.
%
%Visit the \url{http://jmlda.org/?lang=en} website for detailed submission instructions, templates and other information.
%
%Please note that this file must be saved in~\verb'UTF-8' encoding. Where possible select~\verb'UTF-8 without BOM' encoding. 
%To change the encoding please use \verb'Sublime Text' or \verb'Notepad++' text editors.
%
%\section{Structure of the article}
%\noindent
%Divide your article into clearly defined and numbered sections and paragraphs.
%
%\paragraph{Paragraph}
%\noindent
%Sections and paragraphs are numbered and have a brief heading.
%
%%please do not change the name of this section if it is present
%\section{Concluding Remarks}
%This section should provide the summary and explore the significance of the results achieved and list problems not yet solved.
%Results should be clear and concise.

%%%% please specify doi of the cited item if possible, see~\bibitem{article}
%%%% Crossref doi of the item can be retrieved at http://www.crossref.org/guestquery/
\begin{thebibliography}{99}

\bibitem{webArticle} % will need to fix names
	\BibAuthor{Satoshi Tanak, Francois Bonnet, Sebastien Tixeuil, Yasumasa Tamura.} 2020.
	Quixo Is Solved. 
	Available at: \BibUrl{https://arxiv.org/pdf/2007.15895}

\bibitem{webArticle} 
	\BibAuthor{Anian Ruoss, Gregoire Deletang, Sourabh Medapati, Jordi Grau-Moya, Li Kevin Wenliang, Elliot Catt, John Reid and Tim Genewein.} 2024.
	Gradmaster-Level Chess Without Search. 
	Available at: \BibUrl{https://arxiv.org/pdf/2402.04494v1}
	
\bibitem{webArticle} 
	\BibAuthor{Hongyi Guo, Zhihan Liu, Yufeng Zhang, Zhaoran Wang.} 2024.
	Can Large Language Models Play Games? A Case Study of A Self-Play Approach. 
	Available at: \BibUrl{https://arxiv.org/pdf/2402.04494v1}
	
\bibitem{webArticle} 
	\BibAuthor{Lukas Berglund, Asa Cooper Stickland, Meg Tong, Max Kaufmann, Mikita Balesni, Tomasz Korbak, Owain Evans} 2023.
	The Reverasal Curse: LLMs trained on “A is B” fail to learn “B is A”. 
	Available at: \BibUrl{https://arxiv.org/pdf/2206.10498}
	
\bibitem{webArticle} 
	\BibAuthor{Karthik Valmeekam, Matthew Marquez, Alberto Olmo, Sarath Sreedharan, Subbarao Kambhampati.} 2023.
	PlanBench: An Extensible Benchmark for Evaluating Large Language Models on Planning and Reasoning about Change.  
	Available at: \BibUrl{https://arxiv.org/pdf/2206.10498}

\end{thebibliography}

%\maketitleSecondary
%\Russian
%%%% please specify doi of the cited item if possible, see~\bibitem{article}
%%%% Crossref doi of the item can be retrieved at http://www.crossref.org/guestquery/
%\begin{thebibliography}{99}
%\bibitem{book}
%    \BibAuthor{Гуссенс~М., Миттельбах~Ф., Cамарин~А.}
%    \BibTitle{Путеводитель по пакету \LaTeX\ и~его расширению \LaTeXe} / Пер. с англ.~---
%    М.:~Мир, 1999. 606~с.
%    (\BibAuthor{Goossens M., Mittelbach F., Samarin A.}
%     \BibTitle{The \LaTeX\ companion}.~--- 2nd ed.~--- Reading, MA, USA: Addison-Wesley, 1994. 528 p.)
%
%\bibitem{article}
%    \BibAuthor{Загуренко~А.\,Г., Коротовских~В.\,А., Колесников~А.\,А., Тимонов~А.\,В., Кардымов~Д.\,В.}
%    Технико-экономическая оптимизация дизайна гидроразрыва пласта~//
%    \BibJournal{Нефтяное хозяйство}, 2008. Т.~11. \No\,1. С.~54--57.
%	\BibDoi{10.3114/S187007708007}.
%
%\bibitem{webArticle}
%	\BibAuthor{Blaga~P.\,A.}
%	Commutative Diagrams with XY-pic II. Frames and Matrices~//
%	\BibJournal{PracTEX J.}, 2007. Vol.\,4.
%	URL: \BibUrl{https://tug.org/pracjourn/2007-1/blaga/blaga.pdf}.
%
%\bibitem{webResource}
%	XYpic.
%	URL: \BibUrl{http://akagi.ms.u-tokyo.ac.jp/input9.pdf}.
%	
%\bibitem{inproceedingsRus}
%	\BibAuthor{Усманов~Т.\,С., Гусманов~А.\,А., Муллагалин~И.\,З., Мухаметшина~Р.\,Ю., Червякова~А.\,Н., Свешников~А.\,В.}
%	Особенности проектирования разработки месторождений с применением гидроразрыва пласта~//
%	\BibJournal{Труды 6-го Междунар. симп. <<Новые ресурсосберегающие технологии недропользования и повышения нефтегазоотдачи>>}.~---
%	М.:~Издательство, 2007. С.~267--272.
%
%\bibitem{inproceedingsEng}
%    \BibAuthor{Author~N.}
%    Paper title~//
%    \BibJournal{10th Conference (International) on Any Science Proceedings}.~---
%    Place of publication: Publisher, 2009. P.~111--122.
%
%\bibitem{techreport}
%	\BibAuthor{Lambert~P.}
%  	\BibTitle{The title of the work}.
%  	Place of publication:~The institution that published, 1993.  Report~2.
% 	
%\end{thebibliography}


\end{document}
